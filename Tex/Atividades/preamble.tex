\documentclass[a4paper, 12pt]{article} %Tipo de documento.
\usepackage[top=2cm, bottom=2cm, left=2.5cm, right=2.5cm]{geometry} %Pacote para Margens
\usepackage[utf8]{inputenc} % Pacote para permitir a utilização de acentos.
\usepackage{amsmath, amsfonts, amssymb, dsfont} % % Pacote matemático
\usepackage{graphicx} % Pacote inserir imagens.
\usepackage[portuguese]{babel} % Pacote para usar diversos idiomas
\usepackage{float} %Para inserir textos nas imagens
\usepackage{tikz} % Pacote para gráficos Tikz
\usepackage{tabu} %Pacote para tabular
\usetikzlibrary{arrows} % Pacote para o Tikz
\usepackage{setspace} %Pacote para espaçamento entre linhas
\usepackage{cases} % Para definir ordens ao modelo matemático
\usepackage[normalem]{ulem} % Pacote para usar o uuline (texto sublinhado)
\usepackage{multirow} % Pacote para criação de Tabular.
\usepackage{tikz-3dplot}